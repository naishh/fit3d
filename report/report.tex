\section{Skyline detection}
\cite{9} yields a good introduction af the different applied skyline detection
techiques.

In \cite{1}, a column based approach is used. This method is suitable for our
application.

First the contrast of the image is maximized, for now this is done by hand with
GIMP.  Then the image undertakes a Gaussian blur.  On this image a
\textit{sobel} edge detector with a manual threshold is applied.
The result is feed to the Skyline Detector.

The Skyline Detector uses an assumption: the first sharp edge (seen from top to
bottom) is always a skyline/building edge.  It works as follows:
Every column (width:1px) of the edge images is analysed. The value of a pixel
in a column is checked from top to bottom. The system classifies the first
pixel that is above a certain threshold as a skyline pixel.

[1] Castano, Automatic detection of dust devils and clouds on Mars
[9] Cozman, Outdoor visual position estimation for planetary rovers.
