\documentclass[10pt]{article}

%functions
\setlength{\parindent}{0in}

\title{\sc Optimizing 3D models from 2D images}

\author{T. Kostelijk\\mailtjerk@gmail.com}

\date{December 16, 2010}

\begin{document}

\maketitle
\begin{abstract}
Here comes the abstract
\end{abstract}

\section{Introduction}
% TODO Heerlijk de todos zijn niet leesbaar ff ander kleurtje geven ouwe, syntax=tex configgen in vim
% introduction that says we have different images taken from a bulinding etc

\section{Skyline detection}
\textit{ TODO MOTIVATION}
This section describes how the skyline of an image is detected. A skyline
separates a building from the sky and is used as an indicator of the building contour.
\\
% It is interesting to denote that the skyline detector a stand alone method and
% can be optimized individually without any knowledge of the other parts of the
% project.

Some previous work on skyline detection is done.\\
$[9]$ yields a good introduction of different skyline detection techniques.

A comparison is done of the different skyline techniques as described in $[9]$
and $[1]$ is most suitable as a basis for the purpose of this project.  This
method is used as an inspiration. A custom algorithm is made. First
the original method is explained. Then the custom algorithm with respect to the
original algorithm is explained.

The skyline detection algorithm as described in $[1]$ analyzes every column of
the image separately.  The smoothed intensity gradient is calculated from top
to bottom.  The system takes the first pixel with gradient higher then a threshold to be
classified as a skyline element.  This is done for every column in the smoothed
intensity gradient image. The result is a set of coordinates of length $W$,
where $W$ is the width of the image, that represent the skyline.
\textit{Should I elaborate (with a footnote?) on the smoothed
intensity gradient or could I assume this as common knowledge?}. 

Taking the smoothed intensity gradient is the most basic method of edge detection
and has a disadvantage. The method will not be robust to more vague edges.
Instead of the smoothed intensity gradient in this thesis a smoothed edge intensity image is
used.  A practical study is done on the different Matlab build in edge
detection techniques. The Sobel edge detector came with the most promising
results.  Details on this study lie without the bound of this thesis.\\

As a purpose of making the algorithm more precise, two other preprocessing
steps are introduced. First the contrast of the image is increased, this makes
sharp edges stand out more.  Secondly the image undertakes a Gaussian blur,
this removes a large part of the noise.

The system has no several parameters which has to be set manually by the user:
\begin{itemize}
\item contrast parameter, 
\item intensity (window size) of Gaussian blur,
\item Sobel edge detector threshold
\item column based inlier threshold
\end{itemize}
\textit{Should I write down what parameter values I used or is this of too much
detail}

(Automatic parameter estimation based on the image would be interesting future
work but lies without the scope of this research.)

The column based approach from $[1]$ seem to be very useful and is therefor
used as described above but on the preprocessed image.

Some results from the Floriande dataset:
\textit{ TODO Results images}
\textit{ TODO make UML scheme skyline -> 3d -> etc.}
\\
The system assumes that the first sharp edge (seen from top to bottom) is
always a skyline/building edge. This gives raise to some outliers, these are
for example a streetlight or a tree. The outliers are removed as described in
the next section.  The Skyline detector without outlier removal has an
accuracy of (ABOUT) 90 \% (TODO CHECK THIS PERCENTAGE) 
% this is not true, i do not do outtlier removal yet!


\section{Skyline in 3D as a contour of the building}
The retrieved skyline is used to update a sparse 3D model of the building.
This section describes how the skyline of the 2D images are used to get the 3D contour of the
building. 

\subsection{Project to 3D space}
\textit{ TODO situation scheme}
Every 2D pixel of an input image presents a 3D point in space. No
information is known about the distance from the 3D point to the camera. What
is known is de 2D location of the pixel, this reduces the possible points in 3D
space to an infinite line.  This line is known and spanned by two 
coordinates:\\ 
\begin{itemize}
	\item The camera center %(camera centers are annotated for every image)
	\item $K'p$, where $K$ is the Calibration matrix of the camera and $p$ is the homogeneous pixel coordinate.
\end{itemize}

\textit{ TODO why K'p,I don't remember the theory behind it and can't find it in Isaac's paper }
For every skyline pixel a line spanned by the above two coordinates is derived.

\subsection{Intersect with building}
The line is reduced to a point in 3D by intersecting it with a rough indication
of the building. This point is used to update the 3D model.\\
To create a rough indication of the building a top-view photograph of the
building is used together with an estimate of the height of the building.\\ 
In order to refine this sparse 3D model we need to know which skyline part belongs to which part of
the 3D model. This is done as follows.\\
The building is first divided into different walls.  Every wall of the building spans a plane. 
As described in the previous section every part (pixel) of the skyline presents an infinite line in 3D.
Intersections are calculated between these infinite lines and the planes of the building walls.\\
\textit{ Isaac, should I put a intersection formula down here or is this trivial?}\\
Because the lines and the planes are both infinite and they have a very low change
of being exactly parallel, the algorithm returns $w$ intersections for every
skylinepixel (where $w$ is the number of walls).\\
Next challenge is to reduce the number of intersection for every skylinepixel
to one. In other words, to determine the wall that has the largest probability of
being responsible for that pixel. This is ofcourse needed to update the 3D
model at the right place.

\subsection{Find most likely wall}
\subsubsection{When is a wall responsible?}
Lets define the intersection between the projected skylinepixel line and the
plane of the wall as intersection point $isp$. And the wall sides as ${w1, w2,
w3, w4 \in W}$. And $d$ as a distance measure which is explained later on.\\ If
we assume that a certain wall was responsible for that pixel, the intersection
(i.e. projected pixel) must lie either\\

\textbf{(1)} Somewhere on the wall 
\\
or
\\
\textbf{(2)} On a small distance $d$ from that wall
(1) is calculated by testing if the pixel lies inside the polygonal
representation of the wall. This is done using the Matlabs buildin in-polygon
algorithm. If this test succeeds we consider $d$ to be 0.\\

(2) Note that this is treated as an inlier because the 3D model is sparse and the height of
the building is estimated. It is calculated as follows: \\
First the distances from $isp$ to four wall sides are calculated.
For every wall the minimum distance is stored.\\
$min_{w\in W} d(isp, w)$\\
This is done for every wall. The wall with the smallest distance is the one that most likely presents the pixel.\\
$arg min_{W \in Walls} ( min_{w\in W} d(isp, w) )$\\
\textit{ do I write this down correctly?}

%threshold..

If there are two (or even more) walls that are classified equally well to
present the pixel (that is if they succeed the in-polygon or have exactly the
same $d$ value) then the nearest wall is selected. The nearest wall is
calculated by taking the wall with the smallest distance from the $isp$ to the
camera center.
\textit{ formula here? or: trivial?}\\
How this intersection point - wall distance $d$ is calculated is explained in
the next section.

\subsubsection{Calculate the intersection point - wall distance}
A wall consists of four corner points. The corner-point pairs that are on a
side of the wall connect line segments, there are four line segments. These
line segments span infinite lines.\\

The intersection point ($isp$) is projected orthogonally on these four lines, a
projected $isp$ is called $isp_{proj}$.
\textit{ todo image?}
\textit{ todo projection formula?}

If $isp$ is close to a wallside, $e(isp, isp_{proj})$ (where $e$ is defined as
the Euclidean distance) is small.  But this doesn't mean that if
$e(isp,isp_{proj})$ is small $isp$ is always close to the wall. In fact there are
some candidates that happen to have a very small $d(isp, isp_{proj})$ but in fact
lie far away from the wall. This is because $isp$ is projected to an infinite
line spanned by the wallside and could be projected far next to the wallside.
\textit{ An example can be seen in figure:}
\textit{ TODO Figure}
Because of this artefact, it is not robust to calculate the perpendicular
projection distance.  Instead $d$ is calculated differently if $isp_{proj}$
doesn't lie on the wallside.  In this case the Euclidean distance between
$isp_{proj}$ and the closest corner-point of the wallside is returned.

Formally:
Let $c1,c2,c3,c4 \in Cornerpoints$ be the corners of a wall.\\
Let $between(a,b,c)$ be a function that returns true if a lies between b and c.\\
if $between(isp_{proj}, c1, c2)$\\
$d = e(isp, isp_{proj})$\\
otherwise\\
$d = min_{c \in Cornerpoints} e(isp, c)$\\
\\
\textit{TODO nice latex code with large \} sign}\\


\subsubsection{Appendix?: Determine whether $isp_{proj}$ lies on or next to a wallsegment}


% todo ref: http://softsurfer.com/Archive/algorithm_0102/algorithm_0102.htm

% todo P -> $P$
To determine whether the $isp_{proj}$ lies on or next to a wallsegment the
projection calculation is skipped and $isp$ is used instead together with a
computational cheap trick.\\
\textit{ todo insert http://softsurfer.com/Archive/algorithm\_0102/Pic\_segment.gif here}\\
First consider Figure ?? , the angles between the segment P0P1 and the vectors P0P and P1P
from the segment endpoints to P. If both angles are equal to or less than
$90^{\circ}$ then the $isp_{proj}$ wil be on the line segment P0 P1.  If not,
the $isp_{proj}$ lies to the left or to the right of segment P according to
whether one of the angles is acute or obtuse.  The angles are acute or obtuse
if the dot product of the vectors involved are respectively positive or
	negative.\\
To summarize: determining in which region the $isp_{proj}$ lies is boiled down to
two dot product calculations with the advantage that the actual projection
calculation can be skipped.
%Again this result determines if the distance should be computed to one of the
%corner points P0 or P1, or as the perpendicular distance to the line L (the
%wallside) itself.


\textit{Todo come back on outlier removal}

\section{Update 3D model}


\section{References}
\begin{itemize}
\item $[1]$ Castano, Automatic detection of dust devils and clouds on Mars.
\item $[9]$ Cozman, Outdoor visual position estimation for planetary rovers.
\end{itemize}

\end{document}


